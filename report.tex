\documentclass[a4paper,12pt]{report}

\usepackage{alltt, fancyvrb, url}
\usepackage{graphicx}
\usepackage[utf8]{inputenc}
\usepackage{hyperref}
\usepackage{float}
\usepackage{listings}

% Questo commentalo se vuoi scrivere in inglese.
\usepackage[italian]{babel}

\usepackage[italian]{cleveref}

\title{\textbf{Traccia 3 - Chatgame}\\Relazione per il Progetto di \\``Programmazione di Reti''}

\author{Mainardi Giosuè Giocondo}
\date{\today}

\begin{document}

\maketitle

\tableofcontents

\chapter{Introduzione}
Lo scopo del progetto consiste nel realizzare un programma in Python per lo sviluppo di un'architettura Client-Server
%
per il supporto di un ``Multiplayer Playing Game''\footnote{\url{https://it.wikipedia.org/wiki/MMORPG}} testuale.

\section{Spiegazione traccia}

Il gioco si basa principalmente sull'arrivo del Client in una stanza. Questo viene accolto dal \textbf{Master}(Server) il quale 
gli assegna un ruolo, se ce ne sono di disponibili. Il Client verrà ora identificato dal suo ruolo.

Dopodichè il \textbf{Master} gli proporrà un menù con 3 opzioni, due delle quali celano una domanda mentre la
terza è l’opzione trabocchetto.Se sceglie l’opzione trabocchetto viene eliminato dal gioco e quindi esce dalla chat.
Se seleziona invece una delle domande e risponde correttamente al quesito acquisisce un punto, in caso contrario perde un punto.

Il gioco ha una durata temporale finita; il giocatore che al
termine del tempo ha acquisito più punti è il vincitore.

\section{Dominio di gioco}
Per dare un senso al gioco e trovare dei ruoli adatti, ho deciso che la chat fosse relativa ad una 
``Nave Pirata'', così che ogni membro della ciurma avrà il suo ruolo (Capitano, Navigatore, ecc.).
%
L'idea iniziale era di fare ad ognuno domande opportune in base al ruolo, ma per semplicità queste sono
uguali per tutti e riferite al corso di ``Programmazione di Reti''.

\chapter{Descrizione}

\section{Indicazioni}

\subsection{Prerequisiti}
Per poter eseguire corettamente i programmi, è necessario aver installato \textit{Python 3}\footnote{\url{https://www.python.org/downloads/}}.
Inoltre occorre posizionarsi nella cartella principale del progetto, ovvero ``/chatgame''.

\subsection{Esecuzione}
Per una prova esemplificativa si può lanciare il Server da una console, il Client1 da un'altra e il Client2
da un'altra ancora.

\textbf{Attenzione!:} È necessario che il Server sia lanciato prima dei Clients.

\begin{itemize}
	\item \textbf{Server}

	\noindent Lo script del Server può essere eseguito su Bash(Linux) con  :
	\begin{lstlisting}[language=bash]
	  $ python3 server.py
	\end{lstlisting}	
	\noindent Su CMD(Windows) invece  :
	\begin{lstlisting}[language=bash]
	  > python server.py
	\end{lstlisting}
	Una volta eseguito il Server farà partire il Timer di gioco e rimarrà in attesa di eventuali connessioni sull’indirizzo 
	locale 127.0.0.1\footnote{Indirizzo 127.0.0.1 = localhost = Indirizzo di Loopback}.
	\begin{figure}[H]
		\centering{}
		\includegraphics[width=\textwidth]{img/server.jpg}
		\caption{Server in attesa di connessioni.}
		\label{img:server}
	\end{figure}

	\item \textbf{Client1}

	\noindent Lo script del Client1 può essere eseguito su Bash(Linux) con :
	\begin{lstlisting}[language=bash]
	  $ python3 client1.py
	\end{lstlisting}
	\noindent Su CMD(Windows) invece  :
	\begin{lstlisting}[language=bash]
	  > python client1.py
	\end{lstlisting}
	\item \textbf{Client2}
	 
	\noindent Lo script del Client2 può essere eseguito su Bash(Linux) con :
	\begin{lstlisting}[language=bash]
	  $ python3 client2.py
	\end{lstlisting}
	\noindent Su CMD(Windows) invece  :
	\begin{lstlisting}[language=bash]
	  > python client2.py
	\end{lstlisting}
	
	Una volta eseguito il Client si connetterà al Server, se presente, ed aprirà la finestra 
	in cui riceverà i messaggi dal Master e dal quale potrà inviare le risposte.
	\begin{figure}[H]
		\centering{}
		\includegraphics[width=\textwidth]{img/client.jpg}
		\caption{Client appena connesso al Server.}
		\label{img:client}
	\end{figure}

\end{itemize}

È possibile che, lanciando lo script del Server, compaia l'eccezione "", questo vuol dire che da poco
è stata terminata l'esecuzione del Server in ascolto sulla stessa porta(\texttt{53000}) allo stesso indirizzo (Indirizzo di Loopback).
Per evitare questo problema basta assicurarsi di aver terminato il processo in ascolto su quella stessa
porta allo stesso indirizzo e aspettare qualche secondo affinchè il sistema operativo liberi la porta.

In alternativa al dover eseguire tutti e 3 gli script in diverse console, si può eseguire lo script :

\textbf{Test}
	\noindent Lo script del Test può essere eseguito con :
	\begin{lstlisting}[language=bash]
	  $ python3 test.py
	\end{lstlisting}	
Questo script mette in esecuzione, ordinatamente, i 3 programmi, ma tutti nella stessa console. Quindi i vari
ouput su terminale verranno visti insieme.

\subsection{Terminazione programma}
I Clients potrebbero essere espulsi dal Server stesso se questi incappano nel trabocchetto.
Ma se un Client volesse terminare, può tranquillamente farlo con il tasto ``Quit''.
Invece per terminare lo script del Server è necessario posizionarsi sulla console in cui questo
è in esecuzione e digitare contemporaneamente ``Ctrl+C'' da tastiera.

\section{Schemi di flusso}

In questa sezione si mostrano schemi che esemplificano ciò che accade durante il programma.

\subsection*{Client - Server}

\begin{figure}[H]
	\centering{}
	\includegraphics[width=\textwidth]{img/client-server.jpg}
	\label{img:client-server}
\end{figure}

\subsection*{Diagramma di Flusso}

\begin{figure}[H]
	\centering{}
	\includegraphics[width=\textwidth]{img/diagramma.jpg}
	\label{img:diagramma}
\end{figure}

\chapter{Implementazione}

\section{Testing automatizzato}
Durante la creazione del modello e prima della costruzione del relativo Controller si è avuto 
il bisogno di considerare l’uso di test per verificare che ciò che era stato fatto fosse corretto
e per controllare che eventuali modifiche intraprese non compromettessero il lavoro fatto.
Per fare questo è stata utilizzata la libreria Junit (versione 5), per rendere i test automatici.
Le parti controllate in questa fase sono state:
\begin{itemize}
	\item Gestione del combattimento tra diverse unità
	\item Schieramento e rimozione di un'unità in una corsia
	\item Movimento delle unità nella corsia
	\item Generazione del Campo (Field) e interazione con le corsie (Lane)
\end{itemize}

Non volendosi cimentare nel test di interfcce grafiche, sono stati eseguiti periodicamente test manuali 
del gioco che hanno accompagnato la fase di sviluppo.

\section{Metodologia di lavoro}
Per il lavoro in team è stato adottato il DVCS Git con un Workflow semplificato che coinsiste nello sviluppo 
dell'applicazione su un branch develop, dal quale poi un membro in caso di necessità creava un 
branch separato "feature-nomesperimentazione" in cui ha poteva lavorare autonomamente in parallelo su una 
funzionalità a lui assegnata, senza interferire significativamente sulla linea di sviluppo principale
da cui periodicamente ne estraeva eventuali modifiche dei compagni.
%
Al termine della corretta implementazione della funzionalità stabilità, si eseguiva un ulteriore pull dal develop
correggendo eventuali conflitti, per poi riunirsi alla linea di sviluppo principale.

In seguito al raggiungimento di ogni significativo traguardo dell'applicazione si è poi fatta una merge 
del branch develop su quello principale, il master.

Questo processo si è iterato fino al completamento delle funzionalità poste come obbiettivo che hanno portato
al termine del lavoro.

Si vuole indicare in questa sezione che tutti e quattro i membri del gruppo hanno lavorato principalmente in ambiente 
Windows. Si è poi testata la funzionalità dell'applicazione in ambiente Linux, più precisamente nella sistema operativo
Ubuntu 20.04.2 LTS.

\subsection*{Giosuè Giocondo Mainardi}
Come primo passo per l'effettivo sviluppo abbiamo fatto uso di schemi UML in cui rappresentavamo le varie interfacce che saremmo
andati a costruire.

Mi sono concentrato su come poter modellare il dominio applicativo secondo quanto stabilito nelle riunioni periodiche con
il gruppo cercando di riuscire ad offrire, alle classi dipendenti dalle miei, ciò che era richiesto.
Per prima cosa mi sono cimentato nella realizzazione di Field e Lane che modellano lo scontro, essendo agli inizi non avevo modo 
di testare quanto fatto se non con dei Test automatizzati. Grazie a questi ho scovato vari bug man mano e ho modificato poi 
le varie implementazioni dei metodi in base a ciò che doveva risultare dai test. Ad esempio, inizialmente le unita’ non 
venivano correttamente rimosse dalle lane, questo succedeva perché nel metodo update rimuovevo dal Set delle unita’, un elemento 
in un ciclo del foreach, quindi il ciclo si interrompeva perché non riusciva a proseguire con l'iterazione.
Dopo poco tempo mi sono ricordato che questo problema era stato affrontato in laboratorio e mi sono andato a cercare la soluzione
che era per l'appunto usare un iteratore per ciclare sulla collezione, così da poter rimuovere senza problemi un oggetto dal Set
anche durante l'iterazione.

Dopo diverso tempo, quando il campo di gioco è stato realizzato nella view, la mia parte di Model ha saputo rendersi già pronta
e funzionante senza alcun bisogno di modifiche, grazie ai vari test eseguiti nel corso dello sviluppo.

Per realizzare la parte grafica del campo di gioco ho seguito quanto deciso con il gruppo cioè di adottare una GridPane e di riempirla
con immagini, che potevano anche essere vuote, la griglia verra’ poi riempita dalle unita’ aggiunte.

Per il tracciamento della posizione delle unita’ nella singola corsia e del punteggio dei giocatori ho deciso di usare una mappa la quale però in caso
di sovrascrittura di una chiave non offre prestazioni ottime, quindi ho deciso di importare in un package utilities i vari Counter,
visti come esempi nel corso delle lezioni, così da poter modificare il valore mappato semplicemente andando a richiamare un increment
Inoltre i Counter mi offrivano anche la possibilità di impostare un limite e di fare un multi incremento, funzionalità che ho 
sfruttato dato che ho associato ad ogni unita’ il numero di passi che ha compiuto. Così che il raggiungimento della fine della 
corsia sia il raggiungimento di un certo numero di passi, indipendetemente dalla direzione percorsa.

Ho cercato sempre di mantenere un codice chiaro e pulito fin da subito, predisponendo la documentazione delle interfacce
quasi immediatamente così da poter fornire agli altri membri una chiara specifica dell'uso della mia parte.

\subsubsection{Classi/Package quasi esclusivamente miei:}
\begin{itemize}
	\item Field e Implementazione
	\item Lane e Implementazione 
	\item Utilities Counter
	\item FieldView e Implementazione
	\item UnitViewType
	\item Converter
	\item Test di Field e Lane
	\item ViewFactory e Implementazione
\end{itemize}

\subsubsection{Classi/Package in cui ho dato un contributo:}
\begin{itemize}
	\item Music e Sound
	\item Controller e Implementazione
	\item Timer
	\item Unit e UnitType
	\item GameView e Implementazione
	\item Unit, Implementazione e UnitType
\end{itemize}

\section{Note di sviluppo}

Inizialmente non sapendo bene come architettare l'applicazione abbiamo sfogliato un po' il repository con i progetti degli 
anni passati, anche se non abbiamo trovato applicazioni che si avvicinavano al nostro particolare tipo di gioco.

\subsection*{Giosuè Giocondo Mainardi}
\subsubsection{Elenco feature avanzate utilizzate:}
\begin{itemize}
	\item Lambda expressions e Stream dove possibile
	\item Optional
	\item Libreria apache.commons per il Pair 
	\item JavaFX
	\item Meccanismo di chaching delle immagini
\end{itemize}

\appendix

\chapter{Esercitazioni di laboratorio}

\section{Mainardi Giosuè Giocondo}

\begin{itemize}
	\item Laboratorio 04: \url{https://virtuale.unibo.it/mod/forum/discuss.php?d=62685#p101504}
	\item Laboratorio 05: \url{https://virtuale.unibo.it/mod/forum/discuss.php?d=62684#p101506}
	\item Laboratorio 06: \url{https://virtuale.unibo.it/mod/forum/discuss.php?d=62579#p100909}
	\item Laboratorio 07: \url{https://virtuale.unibo.it/mod/forum/discuss.php?d=62582#p100911}
	\item Laboratorio 08: \url{https://virtuale.unibo.it/mod/forum/discuss.php?d=63865#p103957}
	\item Laboratorio 09: \url{https://virtuale.unibo.it/mod/forum/discuss.php?d=64639#p104117}
	\item Laboratorio 10: \url{https://virtuale.unibo.it/mod/forum/discuss.php?d=66753#p106596}
\end{itemize}

\end{document}
