\documentclass[a4paper,12pt]{report}

\usepackage{alltt, fancyvrb, url}
\usepackage{graphicx}
\usepackage[utf8]{inputenc}
\usepackage{hyperref}
\usepackage{float}
\usepackage{listings}

% Questo commentalo se vuoi scrivere in inglese.
\usepackage[italian]{babel}

\usepackage[italian]{cleveref}

\title{\textbf{Traccia 3 - Chatgame}\\Relazione per il Progetto di \\``Programmazione di Reti''}

\author{Mainardi Giosuè Giocondo}
\date{\today}

\begin{document}

\maketitle

\tableofcontents

\chapter{Introduzione}
Lo scopo del progetto consiste nel realizzare un programma in Python per lo sviluppo di un'architettura Client-Server
%
per il supporto di un ``Multiplayer Playing Game''\footnote{\url{https://it.wikipedia.org/wiki/MMORPG}} testuale.

\section{Spiegazione traccia}

Il gioco si basa principalmente sull'arrivo del Client in una stanza. Questo viene accolto dal \textbf{Master}(Server) il quale 
gli assegna un ruolo, se ce ne sono di disponibili. Il Client verrà ora identificato dal suo ruolo.

Dopodichè il \textbf{Master} gli proporrà un menù con 3 opzioni, due delle quali celano una domanda mentre la
terza è l’opzione trabocchetto.Se sceglie l’opzione trabocchetto viene eliminato dal gioco e quindi esce dalla chat.
Se seleziona invece una delle domande e risponde correttamente al quesito acquisisce un punto, in caso contrario perde un punto.

Il gioco ha una durata temporale finita; il giocatore che al
termine del tempo ha acquisito più punti è il vincitore.

\section{Dominio di gioco}
Per dare un senso al gioco e trovare dei ruoli adatti, ho deciso che la chat fosse relativa ad una 
``Nave Pirata'', così che ogni membro della ciurma avrà il suo ruolo (Capitano, Navigatore, ecc.).
%
L'idea iniziale era di fare ad ognuno domande opportune in base al ruolo, ma per semplicità queste sono
uguali per tutti e riferite al corso di ``Programmazione di Reti''.

\chapter{Descrizione}

\section{Indicazioni}

\subsection{Prerequisiti}
Per poter eseguire corettamente i programmi, è necessario aver installato \textit{Python 3}\footnote{\url{https://www.python.org/downloads/}}.
Inoltre occorre posizionarsi nella cartella principale del progetto, ovvero ``/chatgame''.

\subsection{Esecuzione}
Per una prova esemplificativa si può lanciare il Server da una console, il Client1 da un'altra e il Client2
da un'altra ancora.

\textbf{Attenzione!:} È necessario che il Server sia lanciato prima dei Clients.

\begin{itemize}
	\item \textbf{Server}

	\noindent Lo script del Server può essere eseguito su Bash(Linux) con  :
	\begin{lstlisting}[language=bash]
	  $ python3 server.py
	\end{lstlisting}	
	\noindent Su CMD(Windows) invece  :
	\begin{lstlisting}[language=bash]
	  > python server.py
	\end{lstlisting}
	Una volta eseguito il Server farà partire il Timer di gioco e rimarrà in attesa di eventuali connessioni sull’indirizzo 
	locale 127.0.0.1\footnote{Indirizzo 127.0.0.1 = localhost = Indirizzo di Loopback}.
	\begin{figure}[H]
		\centering{}
		\includegraphics[width=\textwidth]{img/server.jpg}
		\caption{Server in attesa di connessioni.}
		\label{img:server}
	\end{figure}

	\item \textbf{Client1}

	\noindent Lo script del Client1 può essere eseguito su Bash(Linux) con :
	\begin{lstlisting}[language=bash]
	  $ python3 client1.py
	\end{lstlisting}
	\noindent Su CMD(Windows) invece  :
	\begin{lstlisting}[language=bash]
	  > python client1.py
	\end{lstlisting}
	\item \textbf{Client2}
	 
	\noindent Lo script del Client2 può essere eseguito su Bash(Linux) con :
	\begin{lstlisting}[language=bash]
	  $ python3 client2.py
	\end{lstlisting}
	\noindent Su CMD(Windows) invece  :
	\begin{lstlisting}[language=bash]
	  > python client2.py
	\end{lstlisting}
	
	Una volta eseguito il Client si connetterà al Server, se presente, ed aprirà la finestra 
	in cui riceverà i messaggi dal Master e dal quale potrà inviare le risposte.
	\begin{figure}[H]
		\centering{}
		\includegraphics[width=\textwidth]{img/client.jpg}
		\caption{Client appena connesso al Server.}
		\label{img:client}
	\end{figure}

\end{itemize}

È possibile che, lanciando lo script del Server, compaia l'eccezione "", questo vuol dire che da poco
è stata terminata l'esecuzione del Server in ascolto sulla stessa porta(\texttt{53000}) allo stesso indirizzo (Indirizzo di Loopback).
Per evitare questo problema basta assicurarsi di aver terminato il processo in ascolto su quella stessa
porta allo stesso indirizzo e aspettare qualche secondo affinchè il sistema operativo liberi la porta.

In alternativa al dover eseguire tutti e 3 gli script in diverse console, si può eseguire lo script :

\textbf{Test}
	\noindent Lo script del Test può essere eseguito con :
	\begin{lstlisting}[language=bash]
	  $ python3 test.py
	\end{lstlisting}	
Questo script mette in esecuzione, ordinatamente, i 3 programmi, ma tutti nella stessa console. Quindi i vari
ouput su terminale verranno visti insieme.

\subsection{Terminazione programma}
I Clients potrebbero essere espulsi dal Server stesso se questi incappano nel trabocchetto.
Ma se un Client volesse terminare, può tranquillamente farlo con il tasto ``Quit''.
Invece per terminare lo script del Server è necessario posizionarsi sulla console in cui questo
è in esecuzione e digitare contemporaneamente ``Ctrl+C'' da tastiera.

\section{Schemi di flusso}

In questa sezione si mostrano schemi che esemplificano ciò che accade durante il programma.

\subsection*{Client - Server}

\begin{figure}[H]
	\centering{}
	\includegraphics[width=\textwidth]{img/client-server.jpg}
	\label{img:client-server}
\end{figure}

\subsection*{Diagramma di Flusso}

\begin{figure}[H]
	\centering{}
	\includegraphics[width=\textwidth]{img/diagramma.jpg}
	\label{img:diagramma}
\end{figure}

\chapter{Implementazione}

\section{Struttura Classi}
Ho realizzato le varie classi in modo tale che classi con funzioni in comune usino le stesse da una libreria.
Nello specifico il file "chat\_server.py" contiene il codice essenziale per l'esecuzione del server nella 
funzione principale "server\_func" che prende in ingresso indirizzo e porta con i quali creare il socket.
Stessa cosa è stata fatta per il Client, realizzando un file "chat\_client.py" che potrà essere importato
per poi lanciare il client passando indirizzo e porta in cui si deve collegarsi.

Il file server.py chiama la funzione di "chat\_server.py" passando l'indirizzo 127.0.0.1 e la porta 53000.
Mentre client1.py e client2.py chiamano entrambi "chat\_client.py" passando l'indirizzo e la porta con 
è stato creato il socket del Server.

\section{Funzioni}
\subsection{Server}
\begin{itemize}
	\item accept\_incoming\_connections():
	
	Funzione che accetta i client in entrata, controlla se ci siano ruoli disponibili (altrimenti aspetta
	 un po' e riprova) e per ogni client connesso fa partire un thread per gestirlo.
	\item handle\_client(client): 
	
	Funzione che gestisce il client passato come parametro. Per prima cosa assegna un ruolo al client,
	glielo comunica, avvisa tutti e poi gli propone il quiz. Se il timer scade, la funzione termina.
	Se il client ha pescato il trabocchetto allora questa funzione aspetta un po' e lo espelle.
	\item quiz\ client():
	
	Funzione che invia il quiz al client, chiede la scelta finchè non viene risposto con '1' '2' o '3'
	e se viene preso il trabocchetto allora termina la funzione, altrimenti faccio la domanda pescata.
	\item check\_answer(client, question):
	
	Funzione che controlla se il client risponde bene alla domanda e assegna il punteggio in base al risultato.
	\item broadcast(msg):
	
	Funzione che manda il messaggio in input a tutti i client.
	\item countdown(timer):
	
	Funzione che tiene il timer e stampa ogni secondo su console.
	\item get\_max\_value(data):
	
	Funzione per ottenere una lista delle chiavi di un dizionario che hanno i valori massimi.
	\item get\_winners():
	
	Funzione per avere una lista di stringhe contenente i ruoli dei client vincitori.
	\item get\_options():
	
	Funzione che mi ritorna una lista mescolata contenente due domande e un trabocchetto.
\end{itemize}

\subsection{Client}
\begin{itemize}
	\item receive():
	
	Funzione che prende i messaggi che arrivano sul Socket, li divide in base al separatore
	e stampa su console i messaggi non vuoti.
	\item send(event=None):
	
	Funzione che prende il contenuto della casella di testo, lo manda al socket e lo stampa 
	in chat.
	\item check\_quit(msg):
	
	Funzione che controlla se l'input sia una richiesta di uscita, in tal caso, 
	chiude il socket e la finestra.
	\item on\_closing(event=None):
	
	Funzione invocata per chiudere il Client
\end{itemize}

\section{Strutture Dati}
\subsection{Server}
\begin{itemize}
	\item clients:
	
	Dizionario che mappa ogni client con il suo ruolo.
	\item addresses:
	
	Dizionario che mappa ogni client con il suo indirizzo.
	\item scores:
	
	Dizionario che mappa ogni client con il suo punteggio.
\end{itemize}

\end{document}
